% !TeX spellcheck = en_US
% Find the template online at
% https://github.com/malexmave/supervision-agreement
% Licensed CC0, Max Maaß
\documentclass[a4paper, 11pt, hidelinks]{article}

\usepackage[utf8]{inputenc}
\usepackage[T1]{fontenc}

\usepackage{fullpage} % changes the margin

\usepackage{hyperref}
\usepackage[all]{hypcap} % fix links to float such that they point to the beginning of the float
\usepackage[capitalise,noabbrev,nameinlink]{cleveref}

\usepackage{paralist}

\usepackage{tgpagella} % Enhanced Palatino
\usepackage{xcolor}
\usepackage{tabu}
\usepackage{booktabs}
\usepackage{enumitem}

\linespread{1.05}

\definecolor{seemoo-orange}{RGB}{236,101,0}
\newcommand{\todo}[1]{\textcolor{seemoo-orange}{#1}}

\title{Supervision Agreement}

\date{}

\begin{document}

\maketitle

Dear prospective student,

you have chosen to write your thesis in our group. That's great, we're looking forward to supervising you! But before we can get started, we want to make sure that we agree on how writing a thesis in our group works, to avoid problems down the line. This document will let you know
\begin{itemize}
	\item what you can \textbf{expect from your supervisor},
	\item what we \textbf{expect from you}, and
	\item how to make use of the \textbf{regular meetings} your supervisor
	may offer you.
\end{itemize}
If you have questions on any of the points below, talk to your supervisor to clear them up.

\section*{What you can expect from your supervisor}
Your supervisor is your first point of contact with the group.
They will help you \textbf{define your thesis topic}, and make sure that it has enough potential to fulfill our requirements for a thesis.
Together, you will define the goals that you should strive to achieve in your thesis.
They will also explain our \textbf{thesis grading scheme} to you, and give you links to a number of excellent past theses written in our group, which you can use as a reference on how to write a great thesis.
These theses go above and beyond our requirements for a perfect grade, but show what other people have achieved within the limited amount of time you have for a thesis.

Writing a thesis is scientific work, and you will need to find and read a number of scientific papers.
Your supervisor will start you out with some relevant literature on your topic, which you can use \emph{as a starting point} for your own literature search.
Finally, they will help you come up with a \textbf{structure for your thesis}.
	
It is inevitable that you will encounter problems while writing your thesis - overcoming them is part of what your thesis is about. 
Your supervisor will \textbf{discuss your ideas for solving these problems with you}. 
But remember: it is \emph{your} thesis, so \emph{you} should be the one who comes up with the ideas.
The same holds for other parts of your thesis - we will be happy to discuss your ideas on how to evaluate them, but we will not do all the work for you.

Your supervisor will be happy to \textbf{comment on \emph{parts of} your thesis} and give you feedback on the structure, methodology and style. 
When sending us drafts to review, please let us know what you have changed since the last draft, and what parts you would like to get our feedback on, as we \textbf{cannot read your whole thesis each time}.
We may point out grammar or spelling mistakes that we find, but keep in mind that we are not a professional review service - \textbf{we will not exhaustively proofread your thesis for grammar and spelling}.
We recommend forming writing groups with your fellow students and proofreading each others theses.

%TODO Adapt especially this part to your group
If you need access to \textbf{special soft- or hardware}, let your supervisor know as soon as possible.
If we don’t have it at hand, we may be able to procure it for you, but this may take some time.
Small purchases (of up to 150 Euro) usually take around a week after approval, larger ones may take longer.

[If you have a student working room, add the details here, including any details on how to enter and leave the building at odd hours, in case there is access control.]

[Add any additional services (eMail addresses, VCS repositories, etc) here.]

Your supervisor may offer you a \textbf{regular meeting}. 
The interval and length of these meetings is something you should discuss with them in the beginning of your thesis.
It is common to have weekly or bi-weekly meeting of 30-60 minutes, but your supervisor may handle things differently.
See the section ``meeting your supervisor'' below for details.

It goes without saying that the following things are \textbf{\emph{not} the task of your supervisor}:

\begin{itemize}
	\item \textbf{Giving you all the literature you will need for the thesis}: Your own independent literature research is part of your thesis and will be graded.
	\item \textbf{Solving all of your problems}: We will be happy to discuss your ideas for solving them, but will not solve them for you (independent work will be graded)
	\item \textbf{Discussing every small problem}: Please try to solve your problems yourself before coming to your supervisor.
	\item \textbf{Being available 24/7}: Your supervisor also has other students and their own research to do. If you need to talk to them, either wait until the next meeting or send an eMail. Your supervisor will not always respond to eMails immediately, especially on the weekend or at night. \emph{Do not drop by their office unannounced}, except in urgent cases or when previously agreed.
\end{itemize}

\section*{What we expect from you}
Writing a thesis is a serious project and involves a lot of work, both from you and from your supervisor.
To make sure things run smoothly, we expect a few things from you.

Most importantly, we expect you to \textbf{work independently}.
Writing a thesis, especially a master thesis, involves doing things no one else has done before - we can help you get started on your topic, but in the end, it has to be \emph{your} thesis if you want a good grade.

We expect you to start your thesis with a \textbf{solid foundation of background knowledge} in your topic area. 
Naturally, you will have to learn some things while working on your thesis, but \textbf{writing a thesis is not the right moment to pick up a completely new skill critical to the thesis} (i.e., don’t expect to write a thesis involving OS-level programming without having a solid working knowledge of C).

\textbf{Writing a thesis is a full-time job}.
According to university guidelines, you are supposed to invest \textbf{40 hours per week} into your thesis.
Don’t expect to write a thesis and also pass four other courses or work a full-time job.
If you cannot invest this time, don’t start writing a thesis.
If you are a part-time student or otherwise have limited time to write your thesis, \textbf{let us know in advance} so that we can plan properly.
%TODO This part is specific to Germany / our university
\textbf{If you lose time due to illness, go to a doctor and get an ``Attest''} - the thesis deadline can only be extended by the examination office if there were important reasons outside of your control why you could not work, and for that, you will need to have some proof like the aforementioned ``Attest''.

If you experience problems that will \textbf{significantly delay your original schedule}, tell your supervisor as soon as possible.
They can work with you to create a new plan, and may also be able to provide resources that help you solve the problem.

We strongly recommend \textbf{handing in drafts of your thesis early}.
These early drafts allow us to give you feedback on the structure of your thesis, while more advanced drafts let us comment on the writing style and illustrations.
Talk to your supervisor to agree on a schedule for sending drafts that works for both of you.

%TODO This is highly specific to CS groups using LaTeX templates
We recommend using a VCS to keep track of your thesis document.
This allows you to roll back any changes or broken files.
[If you offer students a VCS for their thesis, add information here.]
\textbf{Backups are your responsibility}.
Ensure that you have a recent copy of your work in a safe place at all times.

[Add information on mid-term talks, if your group has any.]

[Add information on mandatory templates, if your group uses them.]

It goes without saying that \textbf{plagiarism is unacceptable} and will result in a failing grade.
We are required to report any cases of plagiarism to the examination office, where they will be noted in your student record.
Depending on the severity, you may be expelled from university.
Don't risk it, \textbf{always properly cite your sources}.
If in doubt, ask your supervisor.

This policy on plagiarism also covers source code.
If someone has already solved parts of your problem before and released the code, feel free to use it, but make sure to \textbf{always credit your sources}.
Also take note of the license the source code is released under, and any requirements it may contain (such as strong copyleft clauses in GPL-licensed code).
You are responsible for meeting all locensing requirements of code you are (re)using.

Finally, \textbf{stick to agreements with your supervisor} and follow their advice.
Your supervisor may ask you to write up a summary of each meeting and send it to them afterwards.

\section*{Meeting your supervisor}
The regular meetings offered by your supervisor are intended to ensure that your supervisor always has an idea of how your thesis is going, and that you receive regular feedback on your work.
To do this as effectively as possible and protect both your supervisor’s and your own time, please keep the following points in mind (your supervisor may have additional requirements).

At the start of the meeting, give a \textbf{concise, high-level} overview of what happened since the last meeting.
What did you achieve?
What are the open problems, and how do you want to solve them?
Are you still meeting the planned schedule of your thesis, or are you behind?
If you are behind, how do you plan on catching up?
Please note that we are not interested in a detailed half-hour description of your last week.
Keep your summary concise and ideally to \textbf{five minutes or less}, so that you still have time to discuss any open questions you or your supervisor may have.
Although not required, feel free to use slides or other media to illustrate your summary.

If you want to receive feedback on the draft of your thesis, send it to your supervisor \textbf{a few days in advance of the meeting}.
Keep in mind that, depending on the length of the thesis, your supervisor will need a few days to read it, and they may also be away for conference travel or holidays.
If you need feedback on a large draft, ask your supervisor how far in advance you should send it.

If you did not understand something your supervisor said, \textbf{ask}!
Asking early prevents problems stemming from misunderstandings down the line.

At the end of the meeting, you should \textbf{develop a plan of how you will proceed until the next meeting}.
What will you be working on?
Do you need any resources your supervisor needs to get for you (e.g., hardware, computing power, ...)?
Should your supervisor expect a new draft before the meeting?

\textbf{Let your supervisor know in advance if you need a longer meeting than usual} (e.g., because you have a complex question to discuss that does not fit the normal time you agreed on with your supervisor).
Otherwise, your supervisor may have to leave for another meeting before all questions are cleared up.

Finally, \textbf{be respectful of the time of your supervisor}.
They are probably supervising more than one student, and also have their own research and teaching duties.
Don’t take up their time with small questions you can easily solve yourself, and don’t go on tangents on unrelated topics during your meetings.
If this turns out to be a problem, we may ask you to write an agenda for the meeting in advance, to ensure that you have structured your thoughts and keep to the time limit.
This also includes \textbf{arriving on time}.
If you cannot make it in time, send a quick email and let your supervisor know when you will be arriving.

\section*{Handing in your thesis}
[Add information on how to hand in a thesis at your institution, containing both requirements from your institution and special requirements from your group.]

\section*{Concluding Remarks}
Writing a thesis is \textbf{research}.
This means that you are breaking new ground, and not all eventualities can be predicted.
If it turns out that the idea at the core of your thesis does not
work, \textbf{this is still a result}.
Talk to your supervisor as soon as possible to discuss this.
If you get a negative result (i.e., your idea does not work), try to figure out why it does not work, and collect some ideas on what could be changed to make it work.
\textbf{A thesis with a negative result can still get a very good grade}.
If your initial idea does not work, but you have a new and better idea, discuss this with your supervisor.
The scope of your thesis can be changed, if your supervisor agrees.

[Add information on grading process.]

We want you to succeed, and will do what we can to help you, but \textbf{you have to ask for it}.
Use your insights into your topic to detect problems early and ask for help, and everyone will be happier - you will write a better thesis, and we won’t have to scramble to rescue a thesis two weeks before the deadline.
If, on the other hand, you write an excellent thesis, you may be nominated for an award, and together with your supervisor, you may even be able to write a scientific article based on it.

And that’s it! We’re looking forward to seeing what you will be doing for your thesis.
If anything is unclear or you think we have missed something, feel free to let your supervisor know.


\section*{Resources}
To get you started, here are some resources on scientific writing and how to write a successful thesis.

\begin{itemize}
	\item \href{https://thesisguide.org}{thesisguide.org} gives some general tips and pointers on how to write a good computer science thesis.
	\item ``The Craft of Research'' by Wayne C. Booth \emph{et al.} describes how to write a good research text. It covers time planning, structure, style, and many other aspects of writing a good thesis.
	\item ``Writing Science'' by Joshua Schimel focuses on how to write in a well-structured and memorable way. It is probably a better reference for pure writing style than The Craft of Research, but does not cover other parts of the process like time planning.
	\item ``The Elements of Style'' by Strunk and White is a classic style guide for English writing. It is not specific to scientific writing, but will help you write better in general.
	\item \href{https://www.cse.unsw.edu.au/~gernot/style-guide.html}{Gernot's Guide to Technical Writing} contains another set of pointers on how to write good scientific technical texts.
	\item Specifically for VCS systems, \href{https://chris.beams.io/posts/git-commit/}{chris.beams.io} has a guide on writing good commit messages - some supervisors may ask you to follow it.
\end{itemize}

\end{document}
